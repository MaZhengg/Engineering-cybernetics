
% To be included
\chapter{引言}

如果我们所考虑的系统自由度是$1$,因此,只用一个变数$y$就可以记录或描述这个系统的物理状态。把变数$y$取作时间$t$的函数,也就可以描写这个系统在时间过程中的运动状态。为了确定这个运动状态--也就是函数$y(t)$,我们就必须知道这个系统的构造以及它的各个组成部件的特性。具备了关于系统的这些知识之后,再根据物理学的基本定律把这些知识“翻译”成数学的语言,这样我们就得到一个为了计算 $y(t)$ 而建立的方程。这个方程可能是一个积分方程,也可能是一个积分微分方程,但是在绝大多数的情况下,它是一个微分方程,而且是一个常微分方程,因为只有时间$t$是唯一的自变数。

如果微分方程的每一项中最多只含有因变数$y$或者$y$的各阶时间导数的一次方幂,不包含$y$或者它的各阶时间导数的高次方幂,也不包含这些函数的乘积我们就说这个方程是线性的,同时,也就把这个方程所描述的系统称为\underline{线性系统}。反之,我们就说,这个方程是非线性的,同时,把它所描述的系统称为\underline{非线性系统}。更进一步,还可以把所有线性系统分为\underline{常系数线性系统}和\underline{变系数线性系统}两类。如果描述系统状态的线性微分方程的每一项的系数都是常数,我们就把这个系统称为“常系数线性系统”。如果这些系数不全是常数而是时间$t$的函数,我们就把这个系统称为“变系数线性系统”。

从各类微分方程的解的特性来看,以上的分类方法是有道理的。因为,每个系统的运动状态的特性与描述这个系统的微分方程的类型是有密切关系的。不但如此,微分方程的类型还能确定我们可以对系统提出的合理的问题的性质。换句话说,微分方程的类型确定了解决系统的工程问题的正确做法。现在我们就来看一看这种情况。

\section{常系数线性系统}

让我们来讨论一个最简单的系统--一阶系统。也就是说,微分方程是一个一阶的常系数线性方程。如果假定系统本身的特性不受到外界的影响,并且不受到\underline{驱动函数}(也就是外力)的作用,那么,微分方程就可以写作下列形式:
\begin{equation}\label{eq:first_order_LTI}
\frac{dy}{dt}+ky=0
\end{equation}
其中$k$是一个实常数,可以叫作\underline{弹簧常数}。当$y$不随时间变化时,$dy/dt$等于零。根据方程\eqref{eq:first_order_LTI}必定要有$y=0$。因此,系统的平稳状态(或者平衡状态),就相当于$y=0$的状态。

方程\eqref{eq:first_order_LTI}的解是
\begin{equation}\label{eq:solution_of_first_order_LTI}
y=y_0 e^{-kt}
\end{equation}
这里,$y_0$是$y$的初始值,或者说
\begin{equation}
y(0)=y_0
\end{equation}

\begin{wrapfigure}{r}{0.5\textwidth}
    \centering
    \includegraphics[width=0.48\textwidth]{图1.1.png}
    \caption{ }
    \label{fig:1_1}
\end{wrapfigure}
这样,$y$也就是系统的离开平衡状态的初始扰动。对于正的$k$值和负的$k$值,在图\ref{fig:1_1}里画出$k<0$了系统在$t>0$时的运动状态。我们看到,在$k>0$的情况下,$y$随着时间的增加而逐渐减小。当时间无限增大时,$y\rightarrow 0$。因此,对于$k>0$的情形,系统的扰动就会最后消失掉。于是我们就可以说,系统是稳定的。在$k<0$的情况下,系统的运动随着时间的增加而不断地增大,而且不论初始的扰动位移多么微小,系统的扰动都会逐渐增长到非常大的数值,这也就是说,一旦受到扰动,系统就永远不能再回到平衡状态上去了。这样的系统就是\underline{不稳定}的。

对于阶数更高的系统来说,微分方程里含有更高阶的导数。\underline{$n$阶系统}的微分方程就是:
\begin{equation}\label{eq:n-th-order-diff}
\frac{d^ny}{dt^n}+a_{n-1}\frac{d^{n-1}y}{dt^{n-1}}+\cdots+a_0y=0
\end{equation}
对于实际的物理系统而言,各个系数$a_{n-1},…,a_0$。都是实数。在这种情况下,方程\eqref{eq:n-th-order-diff}的解可以写成
\begin{equation}
y=\sum_{i=1}^n y_0^{(i)}e^{a_i t}\sin(\beta_i t + \varphi_i)
\end{equation}

其中$\alpha_i,\beta_i$都是实数并且和系数a-1,…,有关。各个?: 都是相角。而且也和系数 a-1,…,ao有关。这样一来就可以看出:只有当所有的a;都是负数的时候系统的运动才是稳定的。如果某一个“;是正数,扰动就会越来越大,因而系统也就是不稳定的。

从以上的一些例子可以看到:关于常系数线性系统的运动状态,我们可以问一个严格的问题---\underline{系统的稳定性}的问题。不言而喻,在一个工程设计中,通常的要求就是稳定性。只要确定了微分方程的系数,我们就可以答复系统是否稳定的问题。在由方程\eqref{eq:first_order_LTI}所描述的简单的一阶系统的情况中,$k$的符号是唯一的有决定性意义的资料。

\section{变系数线性系统}

如果在所研究的系统中有一个可变化的参数,变动这个参数就可以使系统的平稳状态或平衡状态相应地改变。很自然地就可以想到:描述系统运动状态的微分方程的系数也是这个参数的函数。例如,作用在飞机上的空气动力就是飞机速率的函数,如果飞机的速率由于加速度或减速度而发生改变的话,那么,即使飞机本身的惯性性质保持不变,作用在飞机上的空气动力也还是要改变的。由于这个缘故,如果我们想计算飞机的离开水平飞行路线的扰动运动的话,基本的微分方程就会是一个变系数的方程。

让我们再回到方程\eqref{eq:first_order_LTI}所描述的一阶系统的简单的例子上去。如果弹簧系数$k$是飞机的速率的函数,而且假定飞机有一个不变的加速度$a$,那么,$k$就是速率$u=at$的函数。因此,微分方程就可以写成以下的形式:
\begin{equation}\label{eq:time_vary_para_diff}
\frac{dy}{dt} + k(at)y=0
\end{equation}
这个方程的解就是:
\begin{equation}\label{eq:solution_of_timevary_para_diff}
\log \frac{y}{y_0}=-\frac{1}{a}\int_0^{at}k(\xi)d\xi,
\end{equation}
其中 $y_0$是初始扰动。如果$k$总是正数,那么,$\log (y/y_0)$就总是负数。而且当时间增大的时候,$\log (y/y_0)$这个负数的绝对值也就会越来越大。因此,$y$就永远小于$y_0$,而且最后趋于消失。所以系统是稳定的。如果$k$总是负数,$\log (y/y_0)$就是一个随着时间增大的正数。即使初始扰动非常微小,$y$的数值最后也会变成很大,所以系统就是不稳定的。这样一些系数不改变符号的变系数系统的特性和常系数系统的特性是非常相近的。

然而,有趣味的是\underline{$k$}既有正值也有负值的情形。我们假定$k(at)$先取正值,然后取负值,最后又再取正值。如果以$u_1=a_{t_1}$表示$k$的第一个零点,以$u_2=a_{t_2}$表示第二个零点,那么,依照我们以前的观念来看,在$u_1$到$u_2$的速度范围之内,系统是不稳定的(图\ref{fig:1_2})。设$y_{\min}$是$y$的极小值,$y_{\max}$是$y$的极大值。根据方程\eqref{eq:solution_of_timevary_para_diff}就有:
\begin{equation}\label{eq:solution_of_timevary_para_diff_min}
\log \frac{y_{\min}}{y_0}=-\frac{1}{a}\int_0^{u_1}k(\xi)d\xi,
\end{equation}
以及
\begin{equation}\label{eq:solution_of_timevary_para_diff_max}
\log \frac{y_{\max}}{y_0}=-\frac{1}{a}\int_0^{u_2}k(\xi)d\xi,
\end{equation}

\begin{figure}[h!]
    \centering
    \includegraphics[width=0.48\textwidth]{图1.2.png}
    \caption{ }
    \label{fig:1_2}
\end{figure}
从工程的观点来看,有兴趣的首要的问题就是:$y_{\max}$多大?是不是它已经大到使系统不能正常运转的程度?我们注意到这样一个事实:为了回答以上的问题,除了$k$和$u$的函数关系之外,我们还需要知道两件事。这两件事就是:加速度$a$多大?初始扰动$y$的大小是多少?因为对于固定的$a$值来说,$y_{\max}$和$y_0$成比例。但是更重要的情况是:对于固定的初始扰动来说,我们可以用增大加速度$a$的办法使偏差的极大值$y_{\max}$大大地减小。这个事实可以从方程\eqref{eq:solution_of_timevary_para_diff_max}看出来。这个事实的实际意义就是:如果尽可能迅速地通过“不稳定区域”,就可以使不利的效果减少到最低的程度。

由以上的讨论我们知道,对于一般的变系数线性系统来说,简单地提出这些系统是否稳定的问题是没有明确的意义的。更有意义的问题的提法是:在给定的扰动和给定的外界条件之下,对于一个确定的\underline{准则}(判断标准)来说,这个系统的运行状态是否使人满意?在我们的简单的一阶系统的例子里,正常运行的确定的判断准则就是$y_{\max}$;给定的扰动就是$y_0$;给定的外界条件就是加速度$a$。因此,由于从常系数系统进展到变系数系统,问题的特点就已经大大地改变了。

为了避免发生误解起见,必须指出:以上的讨论只是为了说明常系数线性系统和变系数系统在基本的数学性质上的区别而已,并不是说,在实际的工程问题中对于常系数线性系统只要求它们稳定就够了,而对于这些系统的其他方面的性能(例如,\underline{过渡过程}中的状态、可能发生的最大偏差$y_{\max}$等等)也还是要加以考虑的。同样地,在实际的工程问题里对变系数线性系统提出的问题也可以是多方面的。总之,希望读者不要把实际的工程问题和理论的说明混淆起来。

\section{非线性系统}
如果在方程\eqref{eq:first_order_LTI}所描述的简单的一阶系统里,弹簧系数$k$是扰动量$y$本身的函数,那么微分方程就成为
\begin{equation}\label{eq:nonlinear_first_order_diff}
\frac{dy}{dt} + f(y) = 0,
\end{equation}
其中 $f(y)=k(y)y$。我们看到这个方程是非线性的。方程\eqref{eq:nonlinear_first_order_diff}所描述的系统也就是非线性系统的最简单的例子。把方程\eqref{eq:nonlinear_first_order_diff}积分,就可以用下列的关系式算出方程的解 $y(t)$:
\begin{equation}
t=-\int_{y_0}^y \frac{d\eta}{f(\eta)},
\end{equation}
这里的 $y_0$仍然是初始扰动。

另外一方面,把方程\eqref{eq:nonlinear_first_order_diff}逐次地求导数就得出:
\begin{equation}\label{eq:nonlinear_first_order_repeated_diff}
\left.
\begin{aligned}
&\frac{d^2 y}{dt^2} + \frac{df}{dy} \frac{dy}{dt} = 0, \\
&\frac{d^3 y}{dt^3} + \frac{d^2 f}{dy^2} \left( \frac{dy}{dt} \right)^2 + \frac{df}{dy} \frac{d^2 y}{dt^2} = 0,\\
\cdots\cdots。
\end{aligned}
\right\}
\end{equation}
因此,如果 $y_1$ 是函数 $f(y)$的零点,并且 $f(y)$在$y_1$点是\underline{正则}的,所以,$f(y)$对于$y$的所有阶的导数在$y_1$点都是有限值。我们还可以假定$f(y)$在$y_1$点附近可以写成
\begin{equation}
f(y)=(y-y_1)^m\left[c_m + c_{m+1}(y-y_1) + cdots \right]
\end{equation}
的形状,其中 $m\geq 1$,而且 $c_m \neq 0$。因此,根据\eqref{eq:nonlinear_first_order_diff}和\eqref{eq:nonlinear_first_order_repeated_diff}就得出:
\begin{equation}
\mbox{在}y=y_1\mbox{处}\frac{dy}{dt}=\frac{d^2y}{dt^2}=\frac{d^3y}{dt^3}=\cdots=0。
\end{equation}
这个事实的意思就是:$y$渐近地趋近于$y_1$。事实上,如果$y_0>y_1$,而且$f(y_0)>0$,那么,$y$最后就会变成$y_1$。如果 $y_0<y_1$ 而$f(y_0)<0$,那么当$t\rightarrow\infty$时,$y$还是变为$y_1$。在$f(y)$的其他的零点附近,$y$的运动状态也还是这种形式的(图\ref{fig:1_3})。
\begin{figure}[h!]
    \centering
    \includegraphics[width=0.48\textwidth]{图1.3.png}
    \caption{}
    \label{fig:1_3}
\end{figure}

如果初始扰动$y_0$与$f(y)$的某一个零点相重合的话,那么,以后$y$就保持着这个数值,并不随时间变化。因此,$f(y)$的各个零点都是平衡位置。如果在某一个零点上$df/dy>0$,就像$y_1$点的情形,离开这个平衡位置的微小偏离必定会逐渐消失,因而系统最后还会回到初始状态上去。这样,我们就可以说,对于微小扰动而言,在 $y_1$点系统是稳定的。可是,如果在某一个零点上$df/dy<0$,就像 $y_2$ 点的那种情形,离开这个平衡点的任何一个微小扰动都会使得系统变动到相邻的平衡位置$y_1$或 $y_3$上去。因此,$y_2$是一个不稳定的平衡状态。

我们已经看到,甚至于像方程\eqref{eq:nonlinear_first_order_diff}所描述的这样一个非常简单的非线性系统,它的运动状态已经是很复杂的了。这样的系统可以同时具有稳定性和不稳定性,因此,对于这一类系统一般地提出是否稳定的问题是毫无意义的。与其这样,倒不如对每一个特殊的问题进行个别的考虑。不过,这里应该指出:我们也可以在某种确定的意义下讨论非线性系统的稳定性问题,譬如:系统在李雅普诺夫意义下或其他与此类似的意义下的稳定性问题$^{[2-4]}$。


关于非线性系统运动状态的稳定性问题,读者可以参阅有关的专门文献\footnote{参阅[1],第4章,\S 5,6}$^,$\footnote{因为所有参考文献都是俄文,俄文文本输入有问题,所以参考文献只标记原参考文献序号,想要查找参考文献可以直接查找影印版}。

\section{工程近似的问题}

几乎可以肯定地这样说:只要加以足够精密的分析,任何一个物理系统都是非线性的。我们说某一个实际的物理系统是线性系统,其意思只是说它可以充分精确地用一个线性系统加以近似地代表而已,并且,所谓“充分精确”的意思就是说:实际系统与理想化了的线性系统的差别,对于具体研究的问题来说已经小到无关紧要的程度。只有当具体的条件和具体的要求明确地给定以后,我们才能把一个实际的系统看作线性系统或是非线性系统。在这个问题上并不存在一般所谓的绝对的判断准则。举例来说,如果我们只想研究一个非线性系统在它的某一个稳定平衡点附近的微小扰动运动的最后状态的话,那么,根据李雅普诺夫的关于运动稳定性的第一近似的定理$^{[2~4]}$,在一定的条件下,原来的系统就可以用一个线性系统很好地近似;但是,如果我们的问题是想研究系统的自激振荡的话,那么,就不能把系统的非线性的性质忽略掉,因为那样一来就会把产生自激振荡的物理根源(和数学根源)丢掉了。

以上所说的处理分类问题的原则,对于把线性系统分为常系数系统和变系数系统两类的情形也是适用的。以方程\eqref{eq:first_order_LTI}和方程\eqref{eq:time_vary_para_diff}所描述的两个简单的系统为例:如果加速度$a$非常小,也就是说飞行的速度几乎不变,由方程\eqref{eq:solution_of_timevary_para_diff_min}就可以看出来$y_{\min}$比初始扰动$y_0$小得多,而且这样的$y_{\min}$发生在$t$的数值很大的一个时刻。在一个有限的时间间隔之内,系统\eqref{eq:time_vary_para_diff}的运动状态和$k$是正值的系统\eqref{eq:first_order_LTI}的运动状态是十分相近的。因此,在一定的场合之下,也可以用常系数系统很准确地近似一个变系数系统。

很明显,常系数系统是最容易研究的。很幸运的是:为数很多的工程系统经过\underline{工程近似}的手续之后,都可以看作常系数系统。这也就是为什么在控制与调节的理论中,关于稳定性的这一部分理论特别发达的缘故。事实上,目前的\underline{伺服系统}理论所处理的基本上就是这一类系统\footnote{应当指出:近年来也对于非线性调节系统和非线性伺服系统进行了大量的研究。参阅[5-9]。}。因此,我们也就先从常系数线性系统开始讨论。