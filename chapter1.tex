
% To be included
\chapter{引言}

如果我们所考虑的系统自由度是$1$,因此,只用一个变数$y$就可以记录或描述这个系统的物理状态。把变数$y$取作时间$t$的函数,也就可以描写这个系统在时间过程中的运动状态。为了确定这个运动状态--也就是函数$y(t)$,我们就必须知道这个系统的构造以及它的各个组成部件的特性。具备了关于系统的这些知识之后,再根据物理学的基本定律把这些知识“翻译”成数学的语言,这样我们就得到一个为了计算 $y(t)$ 而建立的方程。这个方程可能是一个积分方程,也可能是一个积分微分方程,但是在绝大多数的情况下,它是一个微分方程,而且是一个常微分方程,因为只有时间$t$是唯一的自变数。

如果微分方程的每一项中最多只含有因变数$y$或者$y$的各阶时间导数的一次方幂,不包含$y$或者它的各阶时间导数的高次方幂,也不包含这些函数的乘积我们就说这个方程是线性的,同时,也就把这个方程所描述的系统称为\underline{线性系统}。反之,我们就说,这个方程是非线性的,同时,把它所描述的系统称为\underline{非线性系统}。更进一步,还可以把所有线性系统分为\underline{常系数线性系统}和\underline{变系数线性系统}两类。如果描述系统状态的线性微分方程的每一项的系数都是常数,我们就把这个系统称为“常系数线性系统”。如果这些系数不全是常数而是时间$t$的函数,我们就把这个系统称为“变系数线性系统”。

从各类微分方程的解的特性来看,以上的分类方法是有道理的。因为,每个系统的运动状态的特性与描述这个系统的微分方程的类型是有密切关系的。不但如此,微分方程的类型还能确定我们可以对系统提出的合理的问题的性质。换句话说,微分方程的类型确定了解决系统的工程问题的正确做法。现在我们就来看一看这种情况。

\section{常系数线性系统}

让我们来讨论一个最简单的系统--一阶系统。也就是说,微分方程是一个一阶的常系数线性方程。如果假定系统本身的特性不受到外界的影响,并且不受到\underline{驱动函数}(也就是外力)的作用,那么,微分方程就可以写作下列形式:
\begin{equation}\label{eq:first-order-LTI}
\frac{dy}{dt}+ky=0
\end{equation}
其中$k$是一个实常数,可以叫作\underline{弹簧常数}。当$y$不随时间变化时,$dy/dt$等于零。根据方程\eqref{eq:first-order-LTI}必定要有$y=0$。因此,系统的平稳状态(或者平衡状态),就相当于$y=0$的状态。

方程\eqref{eq:first-order-LTI}的解是
\begin{equation}\label{eq:solution-of-first-order-LTI}
y=y_0 e^{-kt}
\end{equation}
这里,$y_0$是$y$的初始值,或者说
\begin{equation}
y(0)=y_0
\end{equation}

\begin{wrapfigure}{r}{0.5\textwidth}
    \centering
    \includegraphics[width=0.48\textwidth]{图1.1.png}
    \caption{ }
    \label{fig:1_1}
\end{wrapfigure}
这样,$y$也就是系统的离开平衡状态的初始扰动。对于正的$k$值和负的$k$值,在图\ref{fig:1_1}里画出$k<0$了系统在$t>0$时的运动状态。我们看到,在$k>0$的情况下,$y$随着时间的增加而逐渐减小。当时间无限增大时,$y\rightarrow 0$。因此,对于$k>0$的情形,系统的扰动就会最后消失掉。于是我们就可以说,系统是稳定的。在$k<0$的情况下,系统的运动随着时间的增加而不断地增大,而且不论初始的扰动位移多么微小,系统的扰动都会逐渐增长到非常大的数值,这也就是说,一旦受到扰动,系统就永远不能再回到平衡状态上去了。这样的系统就是\underline{不稳定}的。

对于阶数更高的系统来说,微分方程里含有更高阶的导数。\underline{$n$阶系统}的微分方程就是:
\begin{equation}\label{eq:n-th-order-diff}
\frac{d^ny}{dt^n}+a_{n-1}\frac{d^{n-1}y}{dt^{n-1}}+\cdots+a_0y=0
\end{equation}
对于实际的物理系统而言,各个系数$a_{n-1},…,a_0$。都是实数。在这种情况下,方程\eqref{eq:n-th-order-diff}的解可以写成
\begin{equation}
y=\sum_{i=1}^n y_0^{(i)}e^{a_i t}\sin(\beta_i t + \varphi_i)
\end{equation}

其中$\alpha_i,\beta_i$都是实数并且和系数a-1,…,有关。各个?: 都是相角。而且也和系数 a-1,…,ao有关。这样一来就可以看出:只有当所有的a;都是负数的时候系统的运动才是稳定的。如果某一个“;是正数,扰动就会越来越大,因而系统也就是不稳定的。

从以上的一些例子可以看到:关于常系数线性系统的运动状态,我们可以问一个严格的问题---系统的稳定性的问题。不言而喻,在一个工程设计中,通常的要求就是稳定性。只要确定了微分方程的系数,我们就可以答复系统是否稳定的问题。在由方程(1.1)所描述的简单的一阶系统的情况中,k的符号是唯一的有决定性意义的资料。



