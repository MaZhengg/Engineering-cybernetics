% To be included
\chapter*{原序}	% 文档类会自动将之加入目录并设置天眉

著名的法国物理学家和数学家安培(A.M.Ampère)曾经给关于国务管理的科学取了一个名字--\underline{控制论}(Cybernétique)[安培著:“论科学的哲学”(Essaisur la philosophie des sciences)第二部,1845年,巴黎出版]。安培企图建立这样一门政治科学的庞大计划并没有得到结果,而且,恐怕永远也不会有结果。可是在这些年代中,各国之间的战争却大大地促进了另一个科学部门的发展,这就是关于机械系统与电气系统的控制与操纵的科学。维纳(N.Wiener)就借用安培所创造的名称“控制论”来称呼这门新的科学,然而,这门科学却是对于现代化战争非常重要的。这真是有些讽刺意味的。维纳的\underline{控制论}(Cybernetics)[“控制论---关于动物体和机器的控制与联系的科学”(Cybernetics,or Control and Communication in the Animal and the Machine , John Wiley \& .Sons, Inc. , NewYork,1948)是关于怎样把机械元件与电气元件组合成稳定的并且具有特定的性能的系统的科学。这门新科学的一个非常突出的特点就是完全不考虑能量、热量和效率等因素,可是在其他各门自然科学中这些因素却是十分重要的。控制论所讨论的主要问题是一个系统的各个不同部分之间的相互作用的定性性质,以及整个系统的总的运动状态。

\underline{工程控制论}的目的是研究控制论这门科学中能够直接用在工程上设计被控制系统或被操纵系统的那些部分。因此,通常在关于伺服系统的书里所讨论的那些问题当然都包括在工程控制论的范围之内。但是,工程控制论比伺服系统工程内容更为广泛这一事实,只是二者之间的一个表面的区别,一个更深刻的,因而也是更重要的区别在于:工程控制论是一门\underline{技术科学},而伺服系统工程却是一种\underline{工程实践}。技术科学的目的是把工程实际中所用的许多设计原则加以整理与总结,使之成为理论,因而也就把工程实际的各个不同领域的共同性显示出来,而且也有力地说明一些基本概念的重大作用。简单地说,理论分析是技术科学的主要内容,而且,它常常用到比较高深的数学工具。只要把本书稍微浏览一下就对这个事实更加清楚了。关于系统的部件的详细构造和设计问题(也就是把理论付诸实践的具体问题)在这本书里几乎是不予讨论的。关于元件的具体问题更是根本不谈的。

能不能够把理论从工程实践分出来研究呢?其实,只要看到目前已经存在的各门技术科学以及它们的飞速发展,就会发现这个怀疑简直是完全不必要的。举一个特别的例子来说:流体力学就是一门技术科学,它与空气动力学工程师,水力学工程师,气象学家以及其他在工作中经常利用流体力学的研究结果的人的实践是“分割”开来的。可是,如果没有流体力学家的话,对于超音速流动的了解和利用至少也要大大地推迟。因此,把工程控制论建成一门技术科学的好处就是:工程控制论使我们可能有更广阔的眼界用更系统的方法来观察有关的问题,因而往往可以得到解决旧问题的更有成效的新方法,而且工程控制论还可能揭示新的以前没有看到过的前景。最近若干年以来,控制与导航技术已经有了多方面的发展,所以,确实也很有必要设法用这样一种统观全局的方法来充分地了解与发挥这种新技术的潜在力量。

因此,关于工程控制论的讨论,应该合理地包括科学中对于工程实践可能有用的所有方面。尤其是不应该仅仅由于数学的困难而逃避任何一个问题,其实深入地考虑一下就会发觉,任何一个问题在数学上的困难常常带有很大的人为的性质。只要把问题的提法稍微加以改变,往往就可以使问题的数学困难减轻到进行研究工作的工程师所能处理的程度。因此,本书的数学水平也就是读过数学分析课程的大学生的水平。关于复变数积分,变分法和常微分方程的基本知识是研读这本书所预先需要的。此外,只要比较直观的讲法能够达到目的,我们就不用严密的精巧的数学方法来讨论;所以,以一个专门作具体工作的电子工程师的眼光来看,我们这种做法一定是太“学究气”了;可是,从一个对这门科学有兴趣的数学家的眼光来看,这种做法可能是太“不郑重”了。倘若以上确是仅有的批评,那么,承蒙各方指正之余,笔者将以为,我并没有违背自己写作这本书时的初衷(这句话在原译著上稍有修改,更接近原意)。

在编写本书期间,作者从和他的两位同事的多次交谈中得益很多,因为,这些谈话常常使一些含混之处突然明确起来。这两位先生就是美国加利福尼亚省理工学院(California Institute of Technology)的马勃尔(Frank E. Marble)博士和德普利马(Charles R. Deprima)博士。由于寨尔登杰克梯(Sedat Serdengecti)和温克耳(Ruth1.Winkel)给予的有效帮助,大大地减少了书稿的准备工作。对于以上提到的各位先生,作者谨表示衷心的感谢。


\vspace{1em}
\begin{flushright}\begin{minipage}{0.3 \textwidth}
	\begin{tabular}{c}
		{\kaishu 钱学森} \\
		  
	\end{tabular}
\end{minipage}\end{flushright}
\vspace{1em}

